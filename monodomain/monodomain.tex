\documentclass[format=acmsmall,screen,timestamp=false,a4paper]{acmart}
\pdfoutput=1
\newif\ifarxiv
\arxivtrue


\ifarxiv
\setkeys{acmart.cls}{nonacm}
\settopmatter{printccs=false,printacmref=false}
\else
\setkeys{acmart.cls}{draft}
\fi
\setcopyright{none}
\acmDOI{}

\acmJournal{TOMS}

\usepackage{amsmath}
\usepackage{caption}
\usepackage{subcaption}
%\usepackage{pgfplots}
%\usepackage{pgfplotstable}
%\pgfplotscreateplotcyclelist{mycolor}{
%%  red,every mark/.append style={fill=red},mark=*\\
%  blue,every mark/.append style={fill=blue},mark=*\\
%  orange,every mark/.append style={fill=orange},mark=*\\
%  red,densely dotted, every mark/.append style={fill=red},mark=square*\\
%  blue,densely dotted, every mark/.append style={fill=blue},mark=square*\\
%  orange,densely dotted every mark/.append style={fill=orange},mark=square*\\
%}

\usepackage{listings}
%\usepackage{xspace}
\usepackage{booktabs}
\usepackage{multirow}
\lstloadlanguages{python}

\newcommand\justin[1]{\textbf{\textcolor[rgb]{1,0,0.5}{[Justin: #1]}}}

% Default listing language
\lstset{language=python,
  basicstyle=\footnotesize\ttfamily,
  showspaces=false,
  showtabs=false,
  frame=tb,
  numbers=left,
  xleftmargin=0em,
  xrightmargin=0em}

\DeclareMathOperator{\Div}{div}
\DeclareMathOperator{\curl}{curl}

\newcommand{\R}{\mathbb{R}}
\newcommand{\red}[1]{\textcolor{red}{#1}}
\newcommand\akg[1]{\textbf{\textcolor[rgb]{.5,0,1}{[Andrew: #1]}}}
\newcommand\josh[1]{\textbf{\textcolor[rgb]{0,.5,1}{[Josh: #1]}}}
\newcommand{\calP}{\mathcal{P}}
\newcommand{\calQ}{\mathcal{Q}}
\newcommand{\calS}{\mathcal{S}}
%\newcommand{\hcurl}{\ensuremath{{H}(\curl)}\xspace}
%\newcommand{\hdiv}{\ensuremath{{H}(\Div)}\xspace}
\newcommand{\hcurl}{\ensuremath{{H}(\curl )}}
\newcommand{\hdiv}{\ensuremath{{H}(\Div )}}

\newcommand{\cancel}[1]{}

\title{Monodomain writeup}


\begin{document}

\maketitle

\section{Introduction}

Consider the monodomain equation, where $u$ is the membrane potential, $\sigma$ is the conductivty tensor $I_\text{ion}$ is the current due to flows of ions through channels in the cell membranes, $C_m$ is the specific capacitance of the cell membrane, and $\chi$ is the surface area to volume ratio:

\[ \chi \big(C_m \frac{\partial u}{\partial t} + I_{\text{ion}}(u) \big) = \nabla \cdot \sigma \nabla u. \]

We derive a weak formulation of this as follows:

\begin{align}
    \chi \big(C_m \frac{\partial u}{\partial t} + I_{\text{ion}}(u) \big) &= \nabla \cdot \sigma \nabla u \\
    \chi v \big(C_m \frac{\partial u}{\partial t} + I_{\text{ion}}(u) \big) &= v \nabla \cdot \sigma \nabla u\\
    \int_\Omega \chi v \big(C_m \frac{\partial u}{\partial t} + I_{\text{ion}}(u) \big) dx &= \int_\Omega v \nabla \cdot \sigma \nabla u dx\\
    \int_\Omega \chi v \big(C_m \frac{\partial u}{\partial t} + I_{\text{ion}}(u) \big) dx &= \int_\Omega \nabla \cdot (v \sigma \nabla u) dx - \int_\Omega \nabla v \cdot \sigma \nabla u dx\\
    \int_\Omega \chi v \big(C_m \frac{\partial u}{\partial t} + I_{\text{ion}}(u) \big) dx &= \int_{\partial \Omega} v \sigma \nabla u \cdot n ds - \int_\Omega \nabla v \cdot \sigma \nabla u dx\\
    \int_\Omega \chi v \big(C_m \frac{\partial u}{\partial t} + I_{\text{ion}}(u) \big) dx &= - \int_\Omega  \nabla v \cdot \sigma \nabla u dx
\end{align}

Note that in line $(3)$ to line $(4)$, we use the identity $f (\nabla \cdot g) = \nabla \cdot (fg) - \nabla f \cdot g$, substituting $f = v$ and $g = \sigma \nabla u$.  From line $(4)$ to line $(5)$ we apply the generalized Stokes' Theorem (Gauss Divergence Theorem in multivariable calculus), $\int_\Omega d F = \int_{\partial \Omega} F $.  Finally, from line $(5)$ to line $(6)$, we require one of two conditions.  Either that $v$ is chosen from a test function space such that $v \vert_{\partial \Omega} = 0$ or that $\nabla u \cdot n = 0$.  



\section{Current Questions}

\begin{enumerate}
    \item Does line $(6)$ look correct under either of my two assumptions for the boundary integral to be 0?
    \item What is $I_{\text{ion}}(u)$? \justin{I know from Andrew's paper that this is determined by a system of ODEs -- what is this going to look like?  Or is it better to just approximate this by some appropriate scalar value for now?}
    \item Test function space condition or boundary condition?  \justin{More generally, when I'm deriving weak forms like this, and want to get rid of the boundary integral, is the process to first check if I have a zero boundary condition, and then if not, force my test function space to be something like $H^1_0$?}
    \item What are proper values for $C_m, \sigma$ and $\chi$?
    \item Determine what way to time step for the PDE?  (I.e., does Firedrake have a built in time-stepper, or do I need to set up an ODE solver myself?)
\end{enumerate}

\end{document}

